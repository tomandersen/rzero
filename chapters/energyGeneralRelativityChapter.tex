\documentclass[../rzero]{subfiles}
\begin{document}
\chapter{Energy In General Relativity}\label{energyGeneralRelativityChapter}

\begin{chapquote}{From Nathan Rosen\cite{rosenEnergyUniverse1994}}
``The energy of the universe, including the energy of the matter and that of the gravitational field, is investigated with the help of the Einstein gravitational pseudo-tensor. It is found that the total energy vanishes.''
\end{chapquote}


\section{Einstein had a problem}
It may be that the Einstein theory of General Relativity is missing something! Einstein, from:\cite{08092323EnergyMomentumGravitational} 

\begin{quotation}
	Thus, we are now effectively forced to distinguish between "matter" and "fields", although we can hope that future generations will overcome this dualistic view and replace it with a single concept, as the field theory of our days has tried in vain to do. 
\end{quotation}


Field energy in General Relativity is a mess. My 'fix' is not a normal one, I'm going to abandon covariance. Let me explain.

\section{The Psuedo Tensor(s)}

A great into is this paper by Nikolic\cite{nikolicTrivialSolutionGravitational2014}



And from Baryshev\cite{08092323EnergyMomentumGravitational}:
\begin{quotation}
	Schrodinger (1918) showed that the mathematical object suggested by Einstein in his final general relativity for describing the energy-momentum of the gravity field may be made vanish by a coordinate transformation for the Schwarzschild solution if that solution is transformed to Cartesian coordinates. Bauer (1918) pointed out that Einstein's energy-momentum object, when calculated for a flat space-time but in a  curvilinear system of coordinates, leads to a nonzero result. In other words, can be zero when it should not be, and can be nonzero when it should.
\end{quotation}


\section{Kutchetera}
Tom views this paper as a plea for solving the gravitational energy problem - that if one takes a energy M, and converts it to a gas of photons that the energy doubles (density term the same in T, pressure terms now exist!). This bothered Landau, etc... 

 

I will first look at Kutschera (2003)\cite{Kutschera2003}, \textit{Monopole gravitational waves from relativistic fireballs driving gamma-ray bursts}

Kutschera studies the impact of this pressure - density formulation in General Relativity. He uses the Einstein Equations as given, and then comes up with '
\begin{quotation}
		The gain of a significant amount of active gravitational mass during the formation period is a direct consequence of Whittaker's formula. It is the pressure-generated contribution that grows rapidly and eventually levels off. The other contribution to the gravitational mass is provided by the total energy of the fireball, which, as a conserved quantity, remains unchanged. Before the formation of the fireball this energy is included in the progenitor mass. \textbf{Hence the gravitational mass of the fireball, composed equally of energy density and pressure contributions, is not a conserved quantity.} This has profound consequences as it implies emission of monopole gravitational waves.
\end{quotation} (I added the bolding of one scentence).

So he finds massive amounts of mass created when matter turns into a non equilbrium gas...a strange consequence of the Einstein equations. I'm not sure Kutschera has the correct viewpoint, in taking Einstein's equations at face value here. See this discussion on the Equivalence Principle in Vishwakarma\cite{vishwakarmaEinsteinCriticalPerspective2016}.

This brings us to Tolman's paradox:
\begin{quotation}
	Tolman’s Paradox: A static spherical box has been filled with a gravitating substance of a given mass. If this substance undergoes an internal transformation (e.g. matter and anti-matter turning into radiation) raising the pressure, the active mass in the box would change because of the 3p-term since the energy is conserved. However, such an internal transformation should not affect the mass measured outside the box, say by an orbiting particle obeying Kepler’s third law. In a spherically symmetric field the particle should be oblivious to all spherically symmetric changes inside its orbit, a consequence of the vacuum equations known as Birkhoff’s Theorem [6].
\end{quotation}


On a more general level, people have studied this problem. There is no easy solution. One would like a covariant gravitational energy tensor to be able to bolt onto the Einstein equations, and fix all this, but Einstein, Landau, Wheeler, etc have looked. It seems it cannot be done.  \cite{08092323EnergyMomentumGravitational}. 


\section{A Schwarzschild like solution}
Here is the plan: Take the quasi local energy definition from Brown and York\cite{Brown1993} or Katz\cite{Katz2005}, and try and apply it to a vacuum Schwarzschild - like solution, where the energy of the gravitational field is now included in the mass term, so that the mass enclosed within a given radius $r$ is less than $M$, the mass at a distance. Which is how the gravitational field of any physical object behaves. But instead of matter, we are allowing the energy of the gravitational field to itself have mass. 

It's pretty easy to see that the resulting metric will not satisfy the usual Einstein vacuum equations, the equations will point to a stress energy tensor on the right side. This then is taken to be an instance of some sort of 'psuedo tensor'. 

What is the function M(r) - what is the mass inside a radius r? 
The Katz formula for the gravitational energy density at radius r in isotropic coordinates is 

\begin{equation} \label{energyoutsideR_eGR}
 GM(r)^2/2 \bar r .
\end{equation}

So in this case the mass energy inside r is the overall, distant constant mass M minus the integral of the mass density from $\bar r$ out to infinity. 

\begin{equation} \label{energyinsideR}
 M - GM^2/2 \bar r .
\end{equation}

But we have to remember that the mass $M$ itself is dropping as we move in, so the $M$ in equation (\ref{energyoutsideR_eGR}) is not a constant. 

By Birchoff's theorem, the gravitational field outside of r should be identical to the Scwharshild, so



\end{document}
