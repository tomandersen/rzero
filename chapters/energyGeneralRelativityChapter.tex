\documentclass[../rzero]{subfiles}
\begin{document}
\chapter{Energy In General Relativity}\label{energyGeneralRelativityChapter}

\begin{chapquote}{Eric Weinstein talking to Brian Keating, \textit{Youtube\cite{drbriankeatingEricWeinsteinTheoretical2020}}}
``You guys need more money. You struck the worlds worst licening deal.''
\end{chapquote}


\section{This chapter}
It may be that the Einstein theory of General Relativity is missing something!\cite{08092323EnergyMomentumGravitational}. 

\section{Kutchetera}
Tom views this paper as a plea for solving the gravitational energy problem - that if one takes a energy M, and converts it to a gas of photons that the energy doubles (density term the same in T, pressure terms now exist!). This bothered Landau, etc... 

 

I will first look at Kutschera (2003)\cite{Kutschera2003}, \textit{Monopole gravitational waves from relativistic fireballs driving gamma-ray bursts}

Kutschera studies the impact of this pressure - density formulation in General Relativity. He uses the Einstein Equations as given, and then comes up with '
\begin{quotation}
		The gain of a significant amount of active gravitational mass during the formation period is a direct consequence of Whittaker’s formula. It is the pressure-generated contribution that grows rapidly and eventually levels off. The other contribution to the gravitational mass is provided by the total energy of the fireball, which, as a conserved quantity, remains unchanged. Before the formation of the fireball this energy is included in the progenitor mass. \textbf{Hence the gravitational mass of the fireball, composed equally of energy density and pressure contributions, is not a conserved quantity.} This has profound consequences as it implies emission of monopole gravitational waves.
\end{quotation} (I added the bolding of one scentence).

So he finds massive amounts of mass created when matter turns into a non equilbrium gas...a strange consequence of the Einstein equations. 

On a more general level, people have studied this problem. There is no easy solution. One would like a covariant gravitational energy tensor to be able to bolt onto the Einstein equations, and fix all this, but Einstein, Landau, Wheeler, etc have looked. It seems it cannot be done.  \cite{08092323EnergyMomentumGravitational}. 




\end{document}
