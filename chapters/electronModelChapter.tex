\documentclass[../rzero]{subfiles}
\begin{document}
\chapter{The Electron Model}\label{electronModelChapter}

\begin{chapquote}{Eric Weinstein talking to Brian Keating, \textit{Youtube\cite{drbriankeatingEricWeinsteinTheoretical2020}}}
``You guys need more money. You struck the worlds worst licening deal.''
\end{chapquote}


\section{This chapter}

Basically, the electron model was leaked in chapter \ref{electromagnetismChapter}. What I add here is a model for the generation of de Broglie waves, thoughts on the 'lack of back reaction' in the de Broglie - Bohm quantum mechanics, etc. 

\section{Model - the Gravitational Zitter Electron}
The electron is modelled as a dynamically perturbed uncharged Kerr solution to the Einstein equations. The naked ring singularity of the Kerr solution is found to have tension.  The  Gravitational Zitter Electron (GZE) is then simply this Kerr singularity - an electron model composed of nothing more than general relativity. These long singularities are found to be extremely efficient gravitational wave generators. In contrast to macroscopic matter, they are ideal 'stir sticks'/'stirees'\footnote{Is stiree a word?} of the Riemann manifold. This efficient transfer of energy from/to the Kerr singularity of the GZE to gravitational waves at Compton frequencies results in the emergence of quantum effects through a de Broglie/Bohm pilot wave mechanism. At much higher frequencies of gravitational wave interaction electromagnetic strength forces emerge between these uncharged Kerr solutions, to be identified with the Coulomb force. Thus the electron can thus be modelled with only general relativity. 

\section{Electromagnetism}
I could just copy section \ref{sectionKerrRingEM} here, but you will go back and read it again, right? As a recap, the electron is modelled as an 'uncharged' Kerr ring, with a long, thread-like singluarity, the length of which allows a carrier wave model of electromagnetic interaction. 
\subsection{The good bits}
One of the main problems with electromagnetism is pointed out by Einstein, (I already quoted this at length in section \ref{sectionNotElectric} but I can't resist this part again): Einstein\cite{Einstein1920}
\begin{quotation}
	Firstly the Maxwell-Loretz equations could not explain how the electric charge constituting an electrical elementary particle can exist in equilibrium in spite of the forces of electrostatic repulsion.  
\end{quotation}

and Feynman\cite{Feynman1985} (again!) in more modern language: 
\begin{quotation}
...it is what I would call a dippy process! 
\end{quotation}

What this zitter model of the electron I am proposing achieves is a way out of this 'charge is self repelling goo' problem. An actual mechanism for electromagnetism is promoted, one which does not use electromagnetism to explain itself. 

I think that's a \href{https://tenor.com/view/martha-stewart-toast-good-thing-gif-4272733}{good thing}. 

\subsection{Not so good bits}
\begin{itemize}
	\item{Why is there quantization of charge? I'm not really sure yet, but I'm sure it will come to me any day now.} 
	\item{The goofy thing about charge quantization is that its so accurate. A proton is composed of like eight to a jillion parts, each one having a charge of +1/3, -2/3, and numbers like that, all coming into and out of existence at some horrific rate. I guess an advantage of good old QM is that this charge quantization is just stated as an unquestionable truth, sort of like taxes.}
	\item{There really isn't a fully accurate description of electromagnetism here, it's just some ideas on how one can generate electromagnetism with general relativity.}
\end{itemize}


\section{de Broglie waves}
\section{From Zitterbewegung to Electric Zitter Electrons}
In 1930 Erwin Schrödinger\cite{schrodinger1930} found that solutions of the Dirac equation for relativistic electrons in free space predict a fluctuation at the speed of light of the position of an electron around its median path with an angular frequency of
\begin{equation}
\frac{2m_ec^2}{h} = \num{1.6e21}\quad radians/sec 
\end{equation}
at a radius of
\begin{equation} \label{Kerr_ring_radius}
	radius_{zitter} = J/mc = \hbar/2m_ec = \num{1.93e-13}m
\end{equation}

This zitterbewegung motion can be pictured as a helical path of Compton radius centred on the mean position of the particle. Several researchers have built electron models based on the helical motion of a charged particle. See Hestenes\cite{Hestenes1990}\cite{hestenes1993kinematic}, Maddox\cite{Maddox1987}, and Barut\cite{Barut1984}. Hestenes summarizes after analyzing the Dirac equation\cite{hestenes1993kinematic}:

\begin{quotation}

\textit{This leads to a self-consistent interpretation of the Dirac theory with the following features:
\begin{enumerate}
\item The electron is modeled as a structureless point particle travelling at the speed of light along a helical lightlike trajectory in spacetime.
\item The helical trajectory has a diameter on the order of a Compton wavelength, and a circular frequency on the order of twice the de Broglie frequency $mc^2/\hbar \approx 10^{-21}s.$
\item The helical motion generates electron spin and may be attributed to magnetic self-interaction.
\item  Each solution of the Dirac equation determines an infinite family of such helices and a probability distribution for the electron to be found on any given helix.
\item  The center of curvature for each helix lies on a streamline of the Dirac current.
\end{enumerate}
}

\end{quotation}

The radius and light speed is identical in value to that of the Kerr model as this is the most economical way to generate the spin angular momentum of $\hbar/2$. Still it is remarkable that the Dirac equation and the Kerr equation both point to a similar physical model, one using relativistic quantum mechanics, the other using the full theory of general relativity. Is zitterbewegung a connection between general relativity and quantum mechanics?
The main differences between the helical zitterbewegung charged electron of Hestenes (Hestenes) and the extreme Kerr solution with electron mass and spin (Present work):
\begin{itemize}
\item The Hestenes electron has charge, while the present work generates electromagnetism from general relativistic gravitational wave interaction.
\item The Hestenes model uses a point like electron, where the present work uses a ring. (Though the ring may often be point like as seen below).
\end{itemize}

\section{Tension in the Singularity}
The electron model presented here is the GZE - the gravitational zitter electron, composed only of a Kerr solution embedded into a dynamic, stochastic background of only general relativity geometry . 

The GZE is a ring singularity. This 'singularity thread' is rotating at c, at a determined size and total mass. The equation for tension in a rotating loop L of rope of mass M, which can be found in elementary physics textbooks is: 
\begin{equation}
Tension = ML\omega^2/(2\pi)^2
\end{equation}

For a ring singularity the rotational velocity is the speed of light, so we have then a direct relationship between $\omega$ and L:
\begin{equation}
\omega = 2\pi c/L
\end{equation}
Thus the tension in a Kerr singularity with electron mass and spin is:
\begin{equation}
T = \frac{2(mc^2)^2}{hc}, \quad T_{GZE} = 0.067 \ N
\end{equation}

So the tension in a Kerr singularity is finite, equals the mass per unit length (in geometric units), and quadratic in mass (for different particles). Keep in mind that larger masses also correspond to smaller rings. The calculation is accurate without using a relativistic expression, as it uses net values for mass per unit length and energy, etc. Note that this tension has the same characteristics as a cosmic strings, where as for example Brandenberger states\cite{Brandenberger2014}:
\begin{quotation}
Since cosmic strings are relativistic objects, a straight string is described by one number, namely its mass per unit length $\mu$ which also equals its tension, or equivalently by the dimensionless number $G\mu$, where $G$ is Newton’s gravitational constant (we are using units in which the speed of light is c = 1).
\end{quotation}
This dimensionless number $G\mu$ for an electron is 
\begin{equation}
	G\frac{m_e}{ring \ circumference} = G \frac{m_e^2c}{\pi\hbar} = \num{5.756e-46}
\end{equation}


The fundamental frequency of a tensioned string is given by 
\begin{equation}
f = \frac{\sqrt{\frac{T}{m/L}}}{2L}
\end{equation}
or since $\sqrt{\frac{T}{m/L}} = c$ in our case, 
\begin{equation}
f = \frac{c}{2L} = \frac{m_e c^2}{h} = \num{1.24e20} Hz
\end{equation}



The fundamental frequency is the Compton frequency. Regarding the rotating ring singularity as a tensioned spinning entity allows one to see that the singularity will not be fixed in a circular shape, but rather it will be deformed by any gravitational waves impinging on it.  

The Compton frequency is the de Broglie fundamental frequency and is 1/2 the zitter frequency. See for example Wignall\cite{Wignall1993} on the relationship between de Broglie frequency and matter waves.

These distortions - which will increase the local curvature of the ring, will result in thicker parts of the distorted ring (remember that higher curvature means higher tension and higher mass per unit length). During  bombardment by gravitational waves the circle of the ring singularity will be pushed into a non circular shape. A non circular ring will have quadrupole moments. It is well known that the emission of gravitational wave radiation tends to circularize elliptical orbits.

The GZE will react to incoming gravitational waves by scattering them in many ways, from super radiance to super absorption. The effect is large and direct, as singularities have a strong grip on spacetime. Singularities on a circular track will emanate gravitational energy efficiently. Superradiance on the global structure will take place at the primary de Broglie frequency, as superradiance peaks at that frequency. 

Nakamura, Shibata and Nakao:\cite{Nakamura1993} talk about the efficient radiation of gravitational waves from a naked singularity, in this case a spindle shaped collapse.
\begin{quotation}
	Intuitively at the formation of the singularity, very short wavelength disturbances of space-time will be created. If there is no event horizon, these disturbances may propagate as gravitational waves so that naked singularity may be a strong source of the very short wavelength (i.e., $\lambda \ll M$, where $M$ is the mass of naked singularity) gravitational waves, which suggests that the singularity itself should suffer strong back reaction.
\end{quotation}

\subsection{Quantum forces, the de Broglie double solution and Fermi}


From \textit{de Broglie’s double solution program: 90 years later} by Colin, Durt and Willox\cite{Colin} 
\begin{quotation}
This was highly dissatisfying for de Broglie and in his effort to reinstate the role of the corpuscle in the theory, he started to develop his double solution program (1925-1927), the (first) core idea of which is that there should be two synchronous, coupled, solutions of the wave equation: 
\begin{itemize}
\item a $\psi$-wave, the phase of which has physical significance but the amplitude of which does not, 
\item and a $u$-wave, which is  a solution describing a moving singularity, the singularity corresponding to the particle. 
A trivial example would be $u(t,{\bf x})=\frac{1}{|{\bf x}-{\bf x}(t)|}e^{i S(t,{\bf x})/\hbar}$.

The phase of the $u$-wave is required to be equal to that of the $\psi$-wave and the $u$-wave itself is supposed to provide a description of the physical reality. 
\end{itemize}
\end{quotation}

In the model presented here, the wholescale 'lowest order' vibration rate of the ring at the Compton frequency produces the $\psi$-wave, which is a high-memory wave that feels out the surroundings, and the ring itself is the $u$-wave - riding the $\psi$-wave in a Couder-Bush\cite{Bush2015}\cite{Turton2018} like way, keeping the ring tied to the phase of the Compton $\psi$-wave. This 'walker force' is enormous, and thus looks like a Bohmian 'quantum potential' - something that sets the velocity of the particle. This velocity setting or 'mysterious quantum potential' is thus explained using Couder-Bush like mechanics, with the substrate being Einstein's ether - a substrate so many orders of magnitude more exact and powerful than the silicon oil of the table top analogs. 

\subsection{The Ring Singularity's apparent size:zero}
The 2007 paper by Arcos1 Pereira (KN refers to the Kerr-Neumann solution): \cite{Arcos2007} 

\begin{quotation}
This result is consistent with previous analysis made by some authors [refs],
who pointed out that an external observer is unable to “see” the KN solution
as an extended object, but only as a point-like object.We can then say that the
“particle” concept is validated in the sense that the non-trivial KN structure
is seen, by all observers, as a point-like object. 
\end{quotation}

In other words the 'r' coordinate in the Kerr solutions (any coord system where the distant coordinate system used approaches the Minkowski system), tells observers far away from the object that the radius appears to be zero. This is the solution to the quantum spin puzzle. The main thing that makes quantum spin so 'quantum' is that its classically impossible to fit $\hbar$ into something the size of an electron. The electron is large, but from the point of view of an experimenter who is far away the size is zero. It's a geometric effect for distant observers. We are always distant in some way so measure the size of the electron to be about 0, with a 'quantum spin' that is too large to fit into the measured size of the electron. 

Carter and Israel point out that the set of paths that end in the singularity is of zero measure.

Carter:\cite{Carter1968}
\begin{quotation}
It will be shown that the ringlike curvature singularities in the inner parts of the Kerr fields are comparatively innocuous (they are in fact invisible except in the equatorial direction) in contrast with the all-embracing curvature singularity in the Schwarzschild solution.
\end{quotation}

It is well known that the gyromagnetic anomaly - the factor of two - of the electron is recreated with a Kerr - Newman. See Israel \cite{Israel1970} for some comments on this. 

\subsection{Cross section of the Ring Singularity}
The cross section is the Planck area, for a Kerr singularity of 'any' mass and angular momentum $hbar/2$. 
Cross section calculation is that the cross section of a section of the singularity of length of Shwarscild radius $r_s$ also has a width $r_s$, so the total cross section of the ring is the length of the ring $L = \pi \hbar/m_e c$ times the width:
\begin{equation}
  \sigma = length*width = \frac{\pi \hbar}{m_e c} \frac{2Gm_e}{c^2} = \frac{2\pi G \hbar}{c^3} = 2\pi Planck\ Area
\end{equation}

So the cross section area of all charged particles is the same, as long as the cross section width increases linearly with mass, which seems reasonable. 
The GZE (or muon, tau) has a cross section for gravitational waves of the Planck Area. 

\subsection{Apparent size of the Ring Singularity}
The cross section is measured by waves small enough to probe the microscopic scale of the ring. For measurements on scales larger than this (say larger than 10-21 metres) we need to use the overall geometry of the Kerr solution.  Physicists probe the size of a particle using other particles. Imagine probing the size of the Kerr singularity with uncharged test mass point particles of high energy - what size would they measure? \cite{Carter1968} points out that the cross section of the singularity measured this way is a 'set of zero measure'. Arcos and Pereira argue that high energy interactions will show electrons\cite{Arcos2004a} \cite{Arcos2007}.The Boyer-Lindquist coordinates also show a radius of zero as 

\subsection{Strength of the GZE interaction}
The force on the GZE from another nearby GZE can be modelled by looking at the exchange of gravitational wave energy between two such particles a distance r apart. 

Nakamura, Shibata and Nakao:\cite{Nakamura1993} estimate that the energy released by gravitational waves in a spindle is such that almost all the initial rest mass energy will be radiated as gravitational waves in some small multiple of the light travel time across the spindle. (see section 3). A GZE when perturbed will be in this spindle like shape, and hence the result hold. shaped approximate spindle shape of some multiple of the distance our case this corresponds to the GZE losing energy and gaining it back from a stochastic background of all other GZEs. The GZE exchanges an amount of energy about equal to some fraction of its mass over each Compton time interval. Call this fraction $\alpha$ 

\subsection{The Highly Efficient Singularity}

The GZE model consists of a string singularity that is a Compton radius in size, spinning at the speed of light. For gravitational waves at a frequency greater then the Compton frequency the singularity can be thought of as a straight section of tensioned string, moving at c. 
The action of a passing gravitational wave in an ideal geometry (lined up with to the rotation axis of the GZE) of amplitude h and frequency f is to alternatively stretch and compress a wavelength long section of the singularity. Since the singularity is under a tension (0.067 Newtons for the GZE), this requires energy.  
The amount of energy exchanged per second on the entire ring for a monochromatic wave is by dimensional analysis: 
If we look at one singularity segment of length $\lambda = 1/f$, a wave passing of amplitude $h$ will cause an energy change in the singularity string of $T \lambda h$. This happens at a frequency $f$ repeated about $ r_{GZE}/\lambda $ times per half cycle on the  singularity. (Not $2 \pi r_{GZE} $ from geometry since only the parts of the singularity parallel to the wave will compress/expand).

Thus for the entire ring:
\begin{equation}
E(flow) = 2 T h f r_{GZE}  .
\end{equation}

Since the singularity is travelling at the speed of light perpendicular to the wave vector, there will be also be energy transferred into the ring singularity (or removed) based on delicate frequency and angle effects. In other words the GZE exchanges energy with the surrounding gravitational field in an efficient manner.

For example for a frequency 1000 times that of the Compton frequency, and a gravitational wave amplitude of $h 1x10-23)$, the energy flux works out to $E(flow) 3x106 eV/s)$ This large force integrated over multiple frequencies might be identified with an emergent electromagnetic field - em from gravitation. 

\subsection{The emergent electric potential}


The electric potential is the fundamental level of electric behaviour in this theory. One arrives at an potential phi as above, and one can see by wiggling a single one of these constructions up and down that a field will be created identical to the vector potential of radiation. The electron on the other end feels this vector potential identically to the way a standard model electron feels the potential, and thus we create EM vector potential and all of Maxwell's equations. 
See \cite{Waves2017} - or at least show how a scalar + vector potential is easier to model/think about. Note that \cite{Waves2017} talks about the reality of the ether. Einstein also believed in the reality of the GR ether.\cite{Einstein1920}

\section{Quantum behaviour from memory effects}

Note that a tuning fork rings at the same frequency no matter how you hit it.

Memory effect of the Kerr response to GW wave bombardment. At resonance, we have the deBroglie fundamental frequency, (i.e. Compton) and after being hit with a GW, the Kerr will re-radiate with memory.  Couder - Bush - de Broglie memory. 

Pauli exclusion principle is related to the two available polarizations of a gravitational wave?

So this memory effect combined with energy absorption and re-radiation IS QM.

Kerr ring has two frequency bands. EM band is high frequency exchange in the linear region of the singularity line, while Compton - de Broglie frequency is QM.

The memory effect can be quantified by estimating the Q value of the ring in various scenarios. (Does Q increase with increasing stochastic GW energy).

SEE FEYNMAN $http://www.feynmanlectures.caltech.edu/I_41.html$ equations 41.5, 41.6 Q is discussed. Q might be very high !! 

\begin{equation}
Q = 2 pi \frac{Energy Stored}{Energy Dissipated per cycle}
 \end{equation}
 
 Energy storage of a ring:
 We have tension, etc say a fraction f of energy is added to the ring, it vibrates. Is the only energy loss in GW? I think so, there is no friction or heating loss. So GW emission is....
 
\begin{equation}
Energy\ stored = \frac{1}{2} \mu \omega^2 A^2 \lambda 
 \end{equation}
 
 Since $m_e = \mu \lambda$
 
 \begin{equation}
Energy\ stored = \frac{1}{2} m_e \omega^2 A^2 
 \end{equation}
 

\begin{equation}
	E_{flux} = \frac{\omega^2 c^3}{32\pi G }( h_+^2 + h_\times^2)
\end{equation}

The quantum energy out/in for an electron is 
\begin{equation}
	E_{flux}\ * \ ring\ width\ *\ ring\ length
\end{equation}

The string singularity has a radius of influence of $r_s$ - so $r_s$ is the ring width and $R_e$ is the ring radius.

The dimensionless $h$ is $A/R$ , so we have 

\begin{equation}
	E_{output\ de\ Broglie} = \frac{\omega^2 c^3}{32\pi G }(\frac{A}{R_e})^2 r_s 2 \pi R_e
\end{equation}

Energy output in ONE cycle is simply 

\begin{equation}
	E_{single cycle} = \frac{\omega c^3}{32\pi G }(\frac{A}{R_e})^2 r_s 2 \pi R_e
\end{equation}

The ratio is the Quality Factor for an electron Kerr ring is the ratio of the two equations above

\begin{equation}
	Q_{quality} = \frac{8 \pi}{4} \frac{\hbar \omega}{c^2 m_e} = 1
\end{equation}

That's a long journey to end at 1! A quality factor of one in a sprung device implies optimal damping.

\section{Conclusion}


Many, if not all physical theories have absurdities built into them. Newtonian gravity has its problems as Newton himself pointed out with instantaneous effect transmission, general relativity has its singularities, and quantum field theory has its renormalization problem, which Feynman called in the case of QED a 'dippy process'\cite{Feynman1985}.
   
The singularities and self inconsistencies in quantum field theory are of a different nature from those in general relativity.
 
Trying to calculate the Standard Model (even with out gravity) using QFT gives rise to many self inconsistencies. For instance even the stability of the vacuum is not guaranteed\cite{Degrassi2013}. The infamous ZPF energy calculation error of 124 orders of magnitude is another example of the frailty of present state of quantum field theory. Another way, perhaps a little harsh, of stating the current situation is that the formalism of the Standard Model breaks down of its own accord. \textit{The Standard Model blows up all on its own.}

In general relativity the singularities only cause problems when they are exposed to other fields, such as electromagnetism, or indeed any Standard Model force. If one takes a universe with only general relativity in it, then singularities are well behaved in the sense that they do not destroy the vacuum, or explode/implode all on their own. \textit{General Relativity does not self destruct all on its own.}

Can we construct elementary particles and quantum mechanics out of nothing more than general relativity? We suppose the answer is yes, but there is much more work to be done. 






 





\cite{hestenesZitterbewegungStructureElectrons2020}, \cite{Burinskii2008}

\end{document}
