\documentclass[../rzero]{subfiles}
\begin{document}
\chapter{Monopole Gravitational Waves}\label{monopoleGravitationalWavesChapter}

\begin{chapquote}{Eric Weinstein talking to Brian Keating, \textit{Youtube\cite{drbriankeatingEricWeinsteinTheoretical2020}}}
``You guys need more money. You struck the worlds worst licening deal.''
\end{chapquote}

\section{Monopole Waves}
Monopole Waves are waves that emanate spherically from a source. Due to symmetry, they can really only be pressure, also known as longitudinal waves. The typical example is a spherical speaker (who has spherical speakers?) turned on. The sound (sound is a pressure wave) goes in all directions and each wave crest forms a spherical pattern, moving away from the speaker at the speed of sound. 

\section{They Can't Exist}
One of the most famous theorems of General Relativity is Birkoff's Theorem\cite{Birkhoff1923}. It's clear, the gravitational field outside a spherical mass is always exactly the static Schwarschild field. So with that big word static in there, it seems there cannot be and monopole gravitational waves. Case closed. 

And yet there are some really interesting papers on this exact subject. Kutchetera: lfkjglskfglkfsg

\section{Kutchetera}

\section{Energy argument for their existence}

\subsection{Velocity of Monopole Waves}
	Like $\num{10e4} c$ 

\section{With Einsteins }

\section{Assuming Chapter \ref{energyGeneralRelativityChapter} Works}
	refer heavily back to chapter \ref{energyGeneralRelativityChapter}. Maybe quote some stuff on other theories of gravity having monopole 

\section{Experimental Findings}
Faster than light collapse



\section{Quantum Mechanics from General Relativity}
my theory


\section{Dark matter is quantum mechanics}
	Model is laid out 
\end{document}
