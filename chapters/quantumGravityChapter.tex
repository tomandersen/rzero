\documentclass[../rzero]{subfiles}
\begin{document}
\chapter{Quantum Gravity}\label{quantumGravityChapter}

\begin{chapquote}{\textit{David Byrne}}
``We sing in the darkness

We open our eyes''
\end{chapquote}


\section{It's Simple}
Well - honestly it is! Since in chapter \ref{quantumMechanicsChapter} and the thesis of the work, we posit that everything is built of gravity, we gravity can't be in a superposition. Particles can't either, and we have a Bohmian trajectory like solution....

Take a double slit experiment. You \textit{can}, in theory use gravity to tell which slit the particle went through, and also see a perfect interference pattern.



This results in specific experimental predictions: 
 

Put poster in here... 

\end{document}
