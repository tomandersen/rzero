\documentclass[../rzero]{subfiles}
\begin{document}
\chapter*{Preface}

\section*{About Tom}
I didn't have a choice.

I became interested in physics at an age young enough that I can't really remember any start date. I attended high school in the small lakeside community of Meaford, Ontario, Canada. blah blah.  

\subsection*{Education}
I attended University of Toronto for my undergraduate studies, starting with a cohort of over 100 students in the Physics Specialist program, which the University of Toronto Physics Department had just finshed remodelling after other top departments around the world. It was a disaster - there were maybe 10 people left by the time third year started - and I wasn't one of them. Thinking that maybe I had made a mistake, I tried a half a year of other topics, before deciding that physics could not be removed from me. So I switched to a double major in math and physics. And by fourth year (ok sixth depending on how you count), I was doing well in any course I cared about, which turned out to be some more abstract math courses, my undergrad thesis course (on Bell's Theorem), and General Relativity. Our fourth year courses were mixed with graduate students entering U of T's physics department.  

Trying to attend graduate school was, I was informed, impossible, as my marks were too low. So I again tried to convince myself that physics wasn't for me, and I painted houses for a year and used the savings to travel around the world. I then met the love of my life, Katherine. 

I applied at a few schools for a masters program in physics, and got in at Laurentian University of Sudbury, Ontario. I lucked out on my choice of advisor, Prof Doug Hallman, as he was collaborator in the then proposed Sudbury Neutrino Observatory. At a Sudbury collaboration meeting in 1992, my future PhD supervisor John Simpson stood up and said "We have an emergency - we've been funded!". After my masters, I went on to John's low level lab at the University of Guelph, where I did my PhD on a few things, mainly the water team, where we built a novel Radon detector and the software side, where I poked around in the 'SNOMAN' software, writing some muon tracking software. John was an amazingly smart, generous and kind advisor. When we went to off site meetings he would always buy wine way above my palette, I think trying, and succeeding at, educating us on the finer points of living. He was an honest gentleman. 

\subsection*{Aside}
The Sudbury Neutrino Observatory was a great success, we built a remarkable ten story high detector 2km under ground at the Sudbury nickel mine. The project leader, Art McDonald won the Nobel prize, and the entire collaboration won the Breakthrough Prize in Physics. I even got a nice plaque and a small cheque.

\subsection*{Career}
After my PhD, we were starting to have children, and a post doc just didn't seem the way to go. For me this was the right decision. I instead started a software company with a dear friend, Ted. We built the world's biggest (8 people) astronomy software package, called Starry Night, which made it easy for everyone from ordinary people to scientists to see where everything was in the night sky, to visit planets, etc. I can't resist blowing my own horn at this point:
\begin{quotation}
	"In the first five years or so of both the Spirit and Opportunity Mars rover missions, Dr. Jim Bell (lead scientist in charge of the on-board Panoramic camera, Pancam) and colleagues on the rover science team occasionally used Starry Night Pro to verify the positions of the moons Phobos and Deimos in the Martian sky, given the positions of the rovers on the surface and the dates and times of the intended observations. These predictions allowed both rovers to acquire time-lapse images of these moons, including daytime "solar eclipse" transits of both Phobos and Deimos across the Sun as well as nighttime "lunar eclipse" passes of Phobos entering and emerging from the shadow of Mars."
 
It really was cool that Starry Night was right on!  It was like having a planetarium program made for Martians!   (and you can quote me on that!)
 
Thanks again,
Jim\footnote{Private communication, 2019}.
\end{quotation} 

For a reason having everything to do with the tech stock market bubble of 2000, we ended up selling the entire operation to Space.com. I was not rich, but hey it helped. 



\section*{Why this book}
While working on several software projects, over years I have kept up with the field of quantum foundations, and lately gravitation. I have always had a different vision on the foundations of quantum mechanics than the mainstream physics community, a vision that is frankly easier to keep by being somewhat on the outside of the community.  

Throughout my software career, I published several papers and attended conferences on quantum foundations and General Relativity. It's hard to publish papers, and often even to attend conferences with a busy job (and three wonderful sons). 

Physics is in crisis [refs]. I'm not the only one, of course that thinks this, but it is by no means an accepted fact in academia. My hope of standing back and watching the community leaders guide physics into the next revolution has faded over the past 20 years.  

 


\end{document}
