\documentclass[../rzero]{subfiles}
\begin{document}
\chapter{TL;DR}\label{introChapter}

\begin{chapquote}{Eric Weinstein talking to Brian Keating, \textit{Youtube\cite{drbriankeatingEricWeinsteinTheoretical2020}}}
``You guys need more money. You struck the worlds worst licening deal.''
\end{chapquote}


\section{Everything all at once}
It seems obvious that I should start the book with a chapter on the present state, and what's wrong, etc. But I don't. Instead I'll outline the entire program, like an executive summary. That way, if you're bored you can cut out early and use the saved time to work on a few more eigenvalues or that pickle ball swing. 

\subsection{It's all gravity}
	Really. That's it. 
	
	Everything that we know and care about is just gravity (really Einstein's General Relativity) formed, like clay, into atoms and light. The forces of nature emerge from dynamical phenonemna in General Relativity, like beer forming from malt, hops, yeast and water. 
	
	Here's the equation:\footnote{But see chapter \ref{energyGeneralRelativityChapter}. Damn that fine print!} 
	

\begin{equation}\label{vacuumEquation} 
R_{\mu\nu} = 0 .  
\end{equation}

Looks simple enough, but there is a lot going on behind there, it's really a set of non linear partial differential equations, (math speak for complicated). The freedom in these non linear equations is crucial for building out a (what you are surely thinking at this point is silly) model of our universe from one concept. 

It does have curb appeal though. So IF one could build something like the phenomena of our universe around us with such a simple equation it would be great. I have been waiting decades for someone to start doing just that. No luck, likely because it's a bad idea, but here we go. 





\section{General Relativity}
If the other fields of physics were this smooth, I maybe wouldn't have had to write this book. That's how smooth it is. General Relativity describes how space and time, behaves (or is it behave?). The magic of Einstein was to realize that space and time \textit{could} behave. He broke with Newton.


\section{Newton}
Newton figured out how the planets orbit, and his theory of gravity is amazingly accurate, but perhaps his biggest message - one that still runs underneath all of (non General Relativity) physics - is that spacetime is a perfect, god - given grid, and on that grid, we have forces. That is a major concept in physics even today. Standard physics won't really let General Relativity in the door.  

\begin{quotation}
	He endures always and is present everywhere, and by existing always and everywhere he constitutes duration and space. Since each and every particle of space is always, and each and every indivisible moment of duration is everywhere, certainly the maker and lord of all things will not be never or nowhere … God is one and the same God always and everywhere. He is omnipresent not only virtually but also substantially; for active power cannot subsist without substance. (Newton 1999: 941).
\end{quotation} 

Einstein's special relativity did not change that. 

Theoretical physics went well until about 1980. Theoretical physics then got worse over time. The culprits were the astronomers, who found that 95\% of the world wasn't in the Standard Model, the experimental physicists, who showed that the world really has faster than light effects, high temperature superconductivity. But perhaps the biggest enemy of all was the top end schools of thought themselves. Only a few lines of thought have been permitted at all, and if 10,000 person years of effort are any indication, these directions are not useful. 


After that, two things happened. Firstly, the model of nature that was settled on around that time, the Standard Model, has gone from explaining virtually everything in the world to about 3-5\% of it. Extensions to that theory, primarily, Super Symmetry, String Theory, and Loop Quantum Gravity have proved unfruitful to say the least. \cite{woit}\cite{SmolinTrouble}\cite{SabineLost}. 




Explain the layout:



1. Generally Covariant\footnote{Chapter xxx will try to argue not only that Einsteins ether exists, but is at rest in our universe.}.

2. Has the .


\begin{itemize}
  \item What's right and wrong in present day physics.
  \item General Relativity
  \item Quantum Mechanics
  \item Energy in General Relativity (foundational field)
  \item Emergent Quantum Mechanics
  \item Quantum Gravity
  \item Electromagnetism
  \item Goopy thoughts 
  \item Goodbye for now
\end{itemize}




\end{document}
